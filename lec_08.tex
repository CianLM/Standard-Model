\lecture{8}{08/02/2025}{Higgs Mechanism}

We have seen spontaneously broken symmetries. Today we will see spontaneously broken gauge symmetries. The natural question is: what happens when a gauge symmetry is spontaneously broken?

Consider the simplest case, the Abelian-Higgs model,
\begin{align}
    S = \int \dd{^{4}x} \left( -\frac{1}{4}F_{\mu \nu} F^{\mu \nu} + D_\mu \phi^{\dag} D^{\mu} \phi - \frac{\lambda}{2} \left( \left| \phi \right|^2 - v^2 \right)^2 \right) 
,\end{align}
where $D_\mu = \partial_\mu \phi - i e A_\mu \phi$.

Now the $U\left( 1 \right) $ symmetry is gauged $\phi \to e^{ie \alpha \left( x \right) } \phi$ and $A_\mu \mapsto A_\mu + \partial_\mu \alpha$.

We take $\phi \left( x \right) = e^{i \Theta\left( x \right) } \left( v + \sigma \left( x \right)  \right)$ and then
\begin{align}
    D_\mu \phi = e^{i \Theta} \left( \partial_\mu \sigma + i \left( v + \sigma \right) \left( \partial_\mu \Theta - e A_\mu \right)  \right) 
,\end{align}
and thus
\begin{align}
    S = \int \dd{^{4}x} \left( -\frac{1}{4} F_{\mu \nu} F^{\mu \nu} + \partial_\mu \sigma \partial^{\mu} \sigma - V \left( \sigma \right)  + \left( v + \sigma \right)^2 \left( \partial_\mu \Theta - e A_\mu \right) \left( \partial^{\mu} \Theta - e A^{\mu} \right)  \right) 
,\end{align}
where $V \left( \sigma \right) = \frac{\lambda}{2}\sigma^2 \left( \sigma + 2 v \right)^2$.

\begin{note}
    $\sigma$ is a massive scalar with $m_\sigma^2 = 2 \lambda v^2$. The analogous field in the standard model is the famous Higgs boson. $\Theta$ appears only as $\partial_\mu \Theta - e A_\mu$ and it can be eliminated by a gauge transformation with $\alpha \left( x \right) = - \frac{1}{e}\Theta \left( x \right) $. This is called \emph{unitary gauge}.
\end{note}

This leads to
\begin{align}
    S = \int \dd{^{4}x} \left( -\frac{1}{4} F_{\mu \nu} F^{\mu \nu} + \partial_\mu \sigma \partial^{\mu} \sigma - V\left( \sigma \right)  + e^2 \left( v + \sigma \right)^2 A_\mu A^{\mu} \right) 
.\end{align}

This contains a quadratic term $v^2 A_\mu A^{\mu}$ which is a mass term for the vector field. Thus, we have a mass for the photon given by
\begin{align}
    m_\gamma^2 = 2 e^2v^2
.\end{align}

This is the \emph{Higgs mechanism}. The would be Goldstone boson $\Theta \left( x \right) $ is `\emph{eaten}' by the gauge field $A_\mu$ to give a massive spin 1 particle.

One can think of the gauge degeneracy as giving us a degenerate set of ground states that are all physically equivalent. The massless fluctuation as one moves between these ground states is no longer there as we have gauged it away.

Recall also that a massless spin one particle has two degrees of freedom, whereas a massive spin one particle has three. The extra degree of freedom here is from the $\Theta$ in the scalar sector.

This same phenomenon happens in a superconductor, where $\phi$ described a bound state of two electrons (Cooper pair) with charge $-2e$. One property of a superconductor is that it expels any magnetic field. This is the \emph{Meissner effect}.

The current coupled to the gauge symmetry is
\begin{align}
    J_\mu = -2e i\left( \phi^{\dag} D \phi - \phi D \phi^{\dag}   \right) 
.\end{align}

If we set $\phi = v e^{i \Theta \left( x \right) }$, we see $J_\mu = 4ev^2 \left( \partial_\mu \Theta - 2e A_\mu \right) $ where $2e$ is the Cooper pair charge. Thus
\begin{align}
    J_i = 4 ev^2 \left( \grad \Theta - 2e A_i \right) \implies \left( \grad \times J \right)_i  = -2 \left( 2e \right)^2 v^2 B_i
.\end{align}

This is the London equation (one of them). Comparing this to Amperes law, $\left( \grad \times  B \right)_i  = J_i$, as $\nabla \times  \left( \nabla \times B_i \right) = - grad^2 B_i$ (using $\nabla \cdot B = 0$) we get
\begin{align}
    \nabla^2 B_i = \frac{1}{\lambda^2} B_i
,\end{align}
with $\lambda^2 = \frac{1}{2 \left( 2 ev \right)^2}$ called the penetration depth.

This equation is solved by $B_i - \left( B\left( z \right) ,0,0 \right) $ where $B\left( z \right) = B_0 e^{-\frac{z}{\lambda}}$. Namely, the magnetic field in the superconductor decays exponentially. This is the Meissner effect.

We propose a thought experiment: take two magnetic monopoles with magnetic charge $g = \pm 1$. In the vacuum, $B_i = \frac{g}{4\pi r^2} \hat{r}_i$ however together their potential becomes $V\left( r \right) \sim \frac{g^2}{r}$. Putting them inside a superconductor, we would see a \emph{magnetic flux tube} form. The magnetic flux tube takes constant energy per unit length, which means that the potential grows $V\left( r \right) \sim  r$ with distance. Magnetic monopoles are then \emph{confined} in a superconductor.

This analogy and process extends to quarks in QCD.
