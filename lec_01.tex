\lecture{1}{23/01/2025}{Introduction}

The standard model is based on the gauge group
\begin{align}
    G = U\left( 1 \right) \times  SU \left( 2 \right) \times  SU \left( 3 \right) 
,\end{align}
with $U\left( 1 \right) $ charge called hypercharge, $SU\left( 2 \right) $ mediating the weak force and $SU\left( 3 \right) $ mediating the strong force with electromagnetism hiding inside $U\left( 1 \right) \times  SU \left( 2 \right) $.

These forces are coupled to 15 Weyl fermions that, collectively, we call the electron, neutrino and down quark. Moreover these particles have to come in a group of four. Then mysteriously, the pattern repeats twice over to give three generations. Each generation experiences exactly the same forces.

% table

\begin{table}[h]
    \centering
    %\caption{}
    \label{tab:parts}
    \begin{tabular}{c|cccc}
    Gen 1 & electron & $e$-neutrino & down & up \\
    mass &1 & $10^{-6}$ & 9 & 4 \\
    \midrule
        Gen 2 & muon & $\mu$-neutrino & strange & charm \\
        mass & 207 & $10^{-6}$ & 186 & 2495 \\
    \midrule
    Gen 3& tau & $\tau$-neutrino & bottom & top \\
       mass & 3483 & $10^{-6}$ & 8180 & $3 \times 10^{5}$ \\
       \midrule 
       charge & -1 & 0 & $-\frac{1}{3}$ & $-\frac{2}{3}$ \\
    \end{tabular}
\end{table}

We have no explanation for these masses. It is tied up with how these particles interact with the Higgs boson.

\subsection{Symmetries}

The structure of the standard model is in large part about its symmetries. This involves Lorentz symmetry, gauge symmetries, global symmetries as well as discrete symmetries.

Minkowski space $\R^{1,3}$ has metric $\eta_{\mu \nu} = \text{diag}\left( 1, -1, -1, -1 \right) $. 

Lorentz transformations map $x^{\mu} \to \tensor{\Lambda}{^{\mu}_\nu} x^{\nu}$ with $\Lambda \subset SO\left( 1,3 \right) $ such that
\begin{align}
    \Lambda^{T} \eta \Lambda = \eta
.\end{align}

We write this group element as
\begin{align}
    \Lambda = \exp \left( -\frac{i}{2} \omega_{\mu \nu} \mathcal{M}^{\mu \nu} \right) 
,\end{align}
with $\mathcal{M}^{\mu \nu} = - \mathcal{M}^{\nu \mu}$ as the generators obeying the algebra commutation relations
\begin{align}
    \left[ \mathcal{M}^{\mu \nu},  \mathcal{M}^{\rho \sigma} \right] = i \left( \eta^{\nu \rho} \mathcal{M}^{\mu \sigma} - \eta^{\nu \sigma} \mathcal{M}^{\mu \rho} + \eta^{\mu \sigma}  \mathcal{M}^{\nu \rho} - \eta^{\mu \rho} \mathcal{M}^{\nu \sigma} \right) 
.\end{align}

\begin{example}
    Observe we can take
    \begin{align}
        \mathcal{M}^{01} = i \mqty( 0 & 1 & 0 & 0 \\ 1 & 0 & 0 & 0 \\ 0 & 0 & 0 & 0 \\ 0 & 0 & 0 & 0 ) && \mathcal{M}^{12} = i \mqty( 0 & 0 &0 & 0 \\ 0 & 0 & -1 & 0 \\ 0 & 1 & 0 & 0 \\ 0 & 0 & 0 & 0 )
    .\end{align}
\end{example}

\subsection{Dirac vs Weyl Spinors}

A Dirac spinor $\psi$ is a 4-component object that transforms in the spinor representation of the Lorentz group. Recall the $\gamma$-matrices defined by $\{\gamma^{\mu} , \gamma^{\nu}\} = 2 \eta^{\mu \nu}$. We take the chiral representation in which
\begin{align}
    \gamma^{\mu} = \mqty( 0 & \sigma^{\mu} \\ \overline{\sigma}^{\mu} & 0 ), && \gamma^{5} = \mqty(\mathbb{I}_2 & 0 \\ 0 & -\mathbb{I}_2)
,\end{align}
where $\sigma^{\mu} = \left( \mathbb{I}_2, \sigma^{i} \right) $ and $\overline{\sigma}^{\mu} = \left(\mathbb{I}_2 , - \sigma^{i} \right) $.

We construct Lorentz generators
\begin{align}
    S^{\mu \nu} = \frac{i}{4} \left[ \gamma^{\mu}, \gamma^{\nu} \right] = \mqty( \sigma^{\mu \nu} & 0 \\ 0 & \overline{\sigma}^{\mu \nu} )
,\end{align}
with $\sigma^{\mu \nu} = \frac{i}{4} \left( \sigma^{\mu} \overline{\sigma}^{\nu} - \sigma^{\nu} \overline{\sigma}^{\mu} \right) $ and $\overline{\sigma}^{\mu \nu} = \frac{i}{4} \left( \overline{\sigma}^{\mu} \sigma^{\nu} - \overline{\sigma}^{\nu} \sigma^{\mu} \right) $.

These obey the Lorentz algebra such that
\begin{align}
    \left[ \sigma^{\mu \nu}, \sigma^{\rho \sigma} \right] = i \left( \eta^{\nu \rho} \sigma^{\mu \sigma} - \eta^{\nu \sigma} \sigma^{\mu \rho} + \eta^{\mu \sigma} \sigma^{\nu \rho} - \eta^{\mu \rho} \sigma^{\nu \sigma} \right) 
,\end{align}
and similarly for $\overline{\sigma}^{\mu \nu}$.
\begin{note}
    This is not an irreducible representation of the Lorentz group. It is reducible as we have two block diagonal matrices that form it.
\end{note}

We then notice that as $S^{\mu \nu}$ is block diagonal, we can decompose
\begin{align}
    \psi = \mqty( \psi_L \\ \psi_R )
,\end{align}
where $\psi_L$ and $\psi_R$ are 2-component \textit{Weyl spinors}. These are irreducible and transform under Lorentz as
\begin{align}
    \psi_L \to S \psi_L \text{~with ~} S = \exp \left(-\frac{i}{2} \omega_{\mu \nu} \sigma^{\mu \nu} \right) 
,\end{align}
and similarly for $\psi_R$.






